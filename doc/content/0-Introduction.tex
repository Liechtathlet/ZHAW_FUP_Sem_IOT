% !TeX encoding=utf8
% !TeX spellcheck = de_CH_frami

\chapter{Einleitung}
Diese Arbeit wurde als Seminararbeit zur Vorlesung von funktionalen Programmiersprachen verfasst.
In diesem Kapitel wird die Aufgabenstellungen und Rahmenbedingungen der Arbeit erläutert.

\section{Hintergrund}
Einer immer grösseren Beliebtheit erfreuen sich kleine Alltagsgegenstände welche mit dem Internet verbunden sind. Dieser Bereich wird "`Internet of Things"' (IOT) genannt. Diese Gegenstände sind in der Lage Daten zu erheben und weiterzuleiten. Da es zukünftig voraussichtlich immer mehr IOT Gegenstände geben wird fallen immer mehr Daten an. 
Diese Daten werden wegen ihrer Masse auch Big Data genannt.

In dieser Arbeit soll evaluiert werden wie sich funktionale Programmiersprachen im Bezug auf IOT eignen, um Big Data auszuwerten.


\section{Ziel}


\section{Aufgabenstellung} \label{sec:Aufgabenstellung}
Die freigegebene Aufgabenstellung lautet wie folgt:

\begin{itemize}
\item 
\end{itemize}


\section{Erwartete Resultate} \label{sec:ErwarteteResultate}
Gemäss freigegebener Aufgabenstellung werden folgende Resultate erwartet:

\begin{itemize}
\item 
\end{itemize}


\section{Abgrenzung} \label{sec:Abgrenzung}
Aufgrund des Umfanges der Arbeit und der begrenzten Zeitdauer werden folgende Punkte von der Arbeit abgegrenzt:

\begin{itemize}
\itemBfText{Schnittstellendokumentation}{In dieser Arbeit werden nicht die Schnittstellendokumentationen und -spezifikationen rekonstruiert. Es werden jeweils die relevanten Aspekte betrachtet und hervorgehoben.}

\end{itemize}


\section{Motivation}

\section{Struktur}


\section{Planung} \label{sec:Intro:Planning}
