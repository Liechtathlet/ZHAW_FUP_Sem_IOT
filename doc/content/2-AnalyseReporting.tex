% !TeX encoding=utf8
% !TeX spellcheck = de_CH_frami

\chapter{Sensordaten auswerten und aufbereiten}
Im vorhergegangenen Kapitel wurde beschrieben, wie die Daten mittels Raspberry Pi und F\# gesammelt wurden. Diese Informationen müssen jetzt aufbereitet und dargestellt werden. Dies wird in diesem Kapitel beschrieben.

\section{Verwendete Hardware}

\section{Verwendete Software}
Die Daten werden mittels F\# auf einem Linux Mint (ein Ubuntu Ableger) dargestellt. Dazu muss Mono installiert werden. 

Danach werden die Daten mit F\# Data\footcite{FShaprp_Data_2016-06-17} ausgelesen, mit F\# aufbereitet und mit FSharp.Charting\footcite{FSharp_Charting_2016-06-17} dargestellt.

\subsection{Mono}
Was Mono ist und wie es installiert wird, wurde im \cref{sec:collect:fsharp:variant1:mono} \nameref{sec:collect:fsharp:variant1:mono} beschrieben.

.NET Core wird für F\# Data und FSharp.Charting nicht benötigt. Deshalb entfällt auch die Installation von \gls{acr:DNVM} und \gls{acr:DNX}. 

\subsection{F\# Data}
F\# Data ist eine Library für den Zugriff auf Daten. Unterstützt werden folgende Formate:
\begin{itemize}
\item CSV
\item HTML
\item JSON
\item XML
\end{itemize}

In dieser Arbeit wurde als Datenformat CSV gewählt, weshalb die folgende Erklärung für dieses Format ausgelegt wurde.
Zuerst muss ein Type erstellt werden, welcher für die Zugriff auf die Daten verwendet wird.
\begin{lstlisting}
type Weather = CsvProvider<"http://www.some-weather-service.org/data.csv", ";">
\end{lstlisting}

Dies lädt eine CSV-Datei von der angegebenen Adresse. Es wird erwartet, dass die Erste Zeile die Namen der Spalten enthält. Zum Beispiel:

\begin{lstlisting}
City;Temperature;Rain
London;6 degree celius;25mm/3h
New York;25 degree celius;<1mm/3h
[...]
\end{lstlisting}

Der Zweite Parameter im obigen Code ist ein Semicolon und definiert, welches Zeichen als Trenner der Spalten verwendet wird.

Danach können die Daten geladen werden:
\begin{lstlisting}
let weatherData = SensorData.Load(""http://www.some-weather-service.org/data.csv")
\end{lstlisting}

Die Variable weatherData enthält eine Sequenz von dem zuvor erstellten Typen, über welche nun zugegriffen werden kann.

\begin{lstlisting}
let londonWeather = weatherData.Rows |> Seq.Head
let londonTemperature = londonWeather.Temperature
\end{lstlisting}

\subsection{FSharp.Charting}
FSharp.Charting ist eine Library zur Visualisierung von Daten. 
Es werden verschiedene Charts unterstützt:

\begin{itemize}
\item Candlestick
\item Line
\item JSON
\item XML
\end{itemize}

\section{Dokumentation der Implementation}